\documentclass{article}
\usepackage[utf8]{inputenc}
\usepackage{cleveref}
\usepackage{amssymb}
\crefname{section}{§}{§§}
\Crefname{section}{§}{§§}
\begin{document}
\begin{center}
{\large\textbf{PREFACE TO THE THIRD EDITION}}
\end{center}
\bigskip
In preparing the third edition of Kelland and Tait's \textit{Introduction to Quaternions} I have been guided mainly by two considerations. In the first place, the average mathematical student of to-day attains either at school or in his early college courses a much higher standard than was possible in 1873 when Kelland wrote, or even in 1881, the date of the second edition. It seemed, therefore, desirable to delete many of the very simple geometrical illustrations which formed a large part of the text, indicating their nature by a word, or transferring them as exercises to the end of the appropriate chapter. In this way valuable space has been gained for the discussion of problems more fitted to bring out the power and beuty of the quaternion calculus.

It is right to mention, however, that Chapter I. has been left exactly as Kelland wrote it; and the greater part of Chapter II. is simply reproduced.

The second consideration was the necessity for presenting the main features of Hamilton's great calculus in a brif but yet logically complete form. This has led to the recasting of Chapters III. and IX. In the new Chapters III. and IV. the calculus in its essential features is developed systematically from the definition of a quaternion as the complex number which measures the ratio of two vectors, with the firther assumption that the associative law hols in product combinations. Fromt these two root principles the whole of Hamilton's powerful vector algebra evolves itself simply and naturally. It is hoped that the mode of presentation will remove the difficulty which some have experienced in accepting Hamilton's identification of vector and quadratal versor. O'Brien, hamilton's brilliant contemporary, confessed that the difficulty was to him insurmountable. But the difficulty is really created by the sceptic himself, who fails to see that, so far as the mathematical definition goes, a vector quantity in quaternions has a much wider significance than the step or displacement or velocity by means of which the simple summation principles are first illustrated. The law of vector addition, which is common to all kinds of vectors, including Hamiltonian, determines nothing as to the laws of product combinations. These may be anything we please among vectors, so long as the law of vecotr addition is satisfied. Now it is proved in Chapter III. \S 18, that quadratal versors obey the vector law of addition. They are therefore true vectors; and hence follows, from the \textit{geometrical} point of view, the analytical identification of the vector and quadratal versor. The identification, no doubt, requires every vector (whatever physical quantity it my symbolise) to be subkect analytically to the quadrautal versor laws in product combinations; but this, as Hamilton himself proved, is tantamount to requireing that three or more vectors in product combinations obey the associative law. There is thus perfect consistency throughout.

From the point of view of pure analysis the difficulty mentioned above cannot, of course, present itself. The quaternioin is then a quantity involving four units, which are defined as reproducing themselves in product combinations and as satisfying certain general laws. The mathematical properties of the quaternion being thus established, the utility of the calculus will depend simply upon the mode of interpretation. Thus C.J. Joly, by a new interpretation of the quaternions, has recently developed an interesting reatment of projective geometry. 

In chapter IX. a completely new section has been introduced on dynamical applications. This seemed to be specially called for, inasmuch as vector ideas and notations are now a familiar feature of some of our best modern books on mathematical physics. It is to be hoped that they will become so more and more, and that the powerful Hamiltonian method which develops the ideas and underlies the notation will become equally familiar.

The last foar articles of Chapter IX. have to do with the cheif properties of the remarkable differential operator $\triangledown$. Differentiation in the ordinary sense was excluded from the earlier editions, although the method was implicitly used in the reatment of tangents. It was impossible, however, to give any true idea of the power of quaternions in dynamix without the explicit introduction of differentiation; and this consideration seemed to me to outweigh all considerations based on artificial distinctions as to what is or is not suitable in an elementary book. The mathematical student who is able to appreciate the exquisite beauties of the linear vector function as expounded in Chapter X. will have no difficulty in appreciating the significance of the Nabla.

Tait's very instructive Chapter X. has been left practically untouched. It is the work of a recognised master, and has been a source of inspiration to many students of the subject. As a pupil of both Kelland and Tait, and as a colleague and friend of the latter, I ahve had peculiar pleasure in preparing this third edition of their joint work, and trust that it may draw the mathematical student into an attractive and largely unexplored field of mathematics. Analystically  the quaternion is now known to take its place in the general theory of complex numbers and continuous groups; it is remarkable that it should have provided for the geometry and dynamics of our visible universe a calculus of great power and simplicity.

My thanks are due to Mt. Peter Ross, M.A., for his careful proof-reading of all but the very earliest Chapters.

\medskip
\hspace{\fill} {\large\textbf{C.G.KNOTT}}\hspace{0.2in}
\medskip

\textsc{Edinburg University}

\hspace{0.2in}{\scriptsize\textit{October}, 1903.}

\end{document}